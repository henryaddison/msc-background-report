\chapter{Modelling} 

I have broken
the project down into three main parts: modelling,
implementation and evaluation. Modelling will highlight which the features in the
environment to be simulated are important and which can be ignored
\cite{Sterling2009}. In the implementation phase, the model is transformed in
computer code so that the simulation can be run. Once modelling and
implementation are sufficiently far along that the code is producing
results, then the evaluation stage can begin and results can be
analysed. Although separate they are not independent. For example, difficulties in implementation may require a rethink in the
model, the implementation must include code that stores results for
evaluation and evaluation of results may lead to changes in the model. Most of the progress so far has been on modelling and so I shall focus
on it for this chapter.


\section{Progress} \label{prog:model}
In order to model the environment and agents for the simulation, we
must simplify. In \ref{sop:objuni} simplification was done by
restricting the number of 
objects within the problem domain. It is also necessary to
simplify the objects themselves so that they can be analysed.

\subsection{River}
\begin{itemize}
  \item A crew carrying their boat over Baits Bite Lock happens
    infrequently enough that it can safely be ignored and Baits Bite
    Lock can be treated as the end of the river.
  \item The boathouses are sufficiently close together and near Jesus
    Lock that it is reasonable to merge the boathouses into one and
    treat this boathouse as the start of the river. 
  \item Shall assume there are equally spaced, frequent landmarks
    along both banks of the river.
\end{itemize}

\subsection{Boat}
\begin{itemize}
  \item The boat is the object that we consider to have a location, speed and direction.
  \item To begin with shall assume all boats are rowing eights.
  \item Since boats only move on the river, can restrict to just looking at the 2D cross-section of where boat
    meets water. Shall physically treat boat as a rigid 17m x 7m rectangle.
  \item A boat will have a fixed maximum speed and fixed acceleration
    rates and deceleration rates (to begin with these may even be
    infinite). When a boat is launched these parameters will be
    selected at random from a distribution to be determined.
\end{itemize}

\subsection{Outing Plan}
\begin{itemize}
  \item An outing plan will specify a single desired speed to travel at
    and a total distance to cover before returning to the
    boathouse. Each time a boat is launched these parameters will be
    chosen at random from distribution to be determined.
  \item To begin with shall ignore time constraints.
\end{itemize}

\subsection{Cox}
Let us consider the cox as an autonomous agent whose actions
(performed by the rowers or the rudder) affect the boat he is in. To
model the cox, we shall follow loosely the layered agent-modelling
technique laid out in \cite{Sterling2009}. 

First we look at the
goals a cox may wish to achieve, the constraints placed upon the cox
and the roles a cox may fill.

\begin{description}
  \item[Goals]
    Some goals a that a cox may wish to pursue are:
    \begin{itemize}
      \item Maintain a speed as close as possible to outing speed.
      \item Cover a distance as close as possible to the outing plan.
      \item Avoid hitting the bank.
      \item Avoid colliding with other crews.
      \item When collisions do occur the relative speed of the colliding object should be as low as possible.
      \item Return to boathouse before specified time.
      \item Avoid holding up other boats unnecessarily
    \end{itemize}
    
  \item[Constraints] The cox is not completely free to achieve the
    goals in any possible way. There are the following constraints:
    \begin{itemize}
      \item Boat must sit fully on the river.
      \item Boat cannot overlap with any other boat.
      \item There is a governing body called CUCBC that have laid down some rules for river use. Unlike the previous two hard, physical constraints these rules of the river (like rowing on the right-hand side) can be ignored. It would be interesting to see which of these rules get learnt naturally and which must be included directly if there are to be followed.
    \end{itemize}
    
  \item[Roles]
    During an outing a cox may switch between many different roles or states. In these states the relative importance of goals may change or new subgoals may be added.
    \begin{itemize}
      \item Following river (including not colliding with other boats)
      \item Overtaking
      \item Stopping
      \item Spinning
    \end{itemize}
  \end{description}
  
  Next we shall consider the actions, perceptions and knowledge base available to the cox.
  
  \begin{description}
    \item[Actions] A cox can perform the following actions:
      \begin{itemize}
        \item Cox can ask rowers to speed up or slow down.
        \item Cox can ask rowers to stop rowing and slow the boat down.
        \item Cox can ask rowers to ask to make an emergency stop.
        \item Cox can ask rowers to ask to move the boat backwards if the boat is stationary.
        \item Cox can steer the boat left or right (this is possible both when moving and when stationary).
        \item Shall treat spinning as an atomic action.
        \item Shall ignore the reaction time required to apply any choices
          made by the cox.
        \end{itemize}
    \item[Perceptions] A cox can make observations about objects in
      his line of sight.
      \begin{itemize}
        \item Shall assume the cox has a full 360\textdegree field of
          view.
        \item Shall assume the cox's vision is not blocked by his or other boats.
        \item Shall assume his
        vision is blocked by the river bank which sits sufficiently high
        above the river to make it impossible to see around corners.
        \item Shall assume the cox can guess reasonable well the speed and
        direction of boats within his line of sight (including his own).
        \end{itemize}
    \item[Knowledge Base] Can assume the cox has access to the
      following knowledge:
      \begin{itemize}
        \item Shall assume the cox has a fairly good memory of the river's
          layout so can guess his location reasonably well from the nearby
          landmarks as mentioned in \ref{riversubsec}.
      \end{itemize}
  \end{description}

\section{Plan}

Section \ref{prog:model} shows the state of the current
model. However, there are still some questions that must be
answered.

One open problem at the moment is how to simulate steering. One option would be to implement some basic physics and allowing the cox to set an angular velocity (so the boat will rotate as well as moving forwards). Another option would be to remove the steering aspect altogether and treat the river as a series of lanes that a boat follows and can move between. A third option is to give the boat almost free movement so the coxing agent has to calculate the path it will follow rather than a set of rudder positions.

For a single boat learning to navigate an empty river it will be necessary to decide what speeds it can
move at (both as desired by an outing plan and maximum) and what rates
turning are available to it. 

Later on when more boats are introduced
it might be nice to try and simulate traffic conditions similar to
those found in the real world and
so it will be necessary to produce a model of how frequently boats
launch and the distribution of speeds and outing plans across boats,
along with any dependencies (for example, I would guess that boats
with a slower maximum speed tend to go out less frequently and have
shorter outings when they do go out).

The next step will be to implement and evaluate a simple version of the model. Initially I plan to produce a simulation that contains just one boat navigating the river. 

\subsection{Implementation}
The process of converting
the model into computer code must be broken down into smaller
tasks. This will also include deciding which tools, such as which
language and pre-existing libraries or frameworks, are appropriate 
to use. I have also given a brief description of the manner in which I
plan to work in the Appendix \ref{app:sem}

\subsection{Evaluation}
In order to evaluate any results produced, it is necessary to decide
first what should be considered as successful outcomes and then how
these should be measured. For example, from a safety point of view it
would be desireable to cutdown on dangerous crashes, which might be
measured as the number of collisions above a certain speed.
