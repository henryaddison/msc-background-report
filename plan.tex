\chapter{Progress and Plan}

I have broken
the project down into three main sections - modelling,
implementation and evaluation. Modelling will highlight which the features in the
environment to be simulated are important and which can be ignored
\cite{Sterling2009}. In the implementation phase, the model is transformed in
computer code so that the simulation can be run. Once modelling and
implementation are sufficiently far along that the code is producing
results, then the evaluation stage can begin and results can be
analysed. Although they are discussed separately, they all affect each
other. For example, difficulties in implementation may require a rethink in the
model and the implementation must include code that stores results for
evaluation.

\section{Progress}
Most of the progress so far has been on modelling and so I shall focus
on it for this section. With the model incomplete not much progress
can be made with implementation - I have spent some time playing
around with vtk and NetLogo. And without an implementation to produce
results certainly no progress has been made on evaluation.

\subsection{Modelling} \label{prog:model}
In order to model the environment and agents for the simulation, we
must simplify. In section \ref{sop:objuni} the first step in this
simplification process by restricting the
objects within the problem domain. It is also necessary to
simplify the objects themselves so that they can be analysed.

\subsubsection{River}
\begin{itemize}
  \item A crew carrying their boat over Baits Bite Lock happens infrequently enough that it can safely be ignored.
  \item The boathouses are sufficiently close together and near Jesus Lock that it is reasonable to merge the boathouses into one. 
  \item Outside of race days and a brief period early in the year when
    novice crews take to the water, the short stretch of water from
    Christ's boathouse (the last in the row) to Jesus Lock is rarely
    used and so it seems reasonable to treat this merged boathouse as
    one end point of the river rather than Jesus Lock.
  \item Shall assume there are equally spaced, frequent landmarks
    along both banks of the river.
\end{itemize}

\subsubsection{Boat}
\begin{itemize}
  \item To begin with shall assume all boats are rowing eights.
  \item Restrict to just looking at the 2D cross-section of where boat
    meets water, so can consider a boat to be a rigid 17m x 7m rectangle for
    the purposes of collision detection.
  \item A boat will have a fixed maximum speed and fixed acceleration
    rates and deceleration rates (to begin with these may even be
    infinite). When a boat is launched these parameters will be
    selected at random from a distribution to be determined.
\end{itemize}

\subsubsection{Outing Plan}
\begin{itemize}
  \item An outing plan will specify a single desired speed to travel at
    and a total distance to cover before returning to the
    boathouse. Each time a boat is launched these parameters will be
    chosen at random from distributeion to be determined.
  \item To begin with shall ignore time constraints.
\end{itemize}

\subsubsection{Cox}
Let us consider the cox as an autonomous agent whose actions
(performed by the rowers or the rudder) affect the boat he is in. To
model the cox, we shall follow loosely the layered agent-modelling
technique laid out in \cite{Sterling2009}. We shall also look at the
goals a cox may wish to achieve, the various roles a cox may fulfill
and the constraints placed upon the cox. We shall also consider the
actions, perceptions and knowledge base available to the cox.

\begin{itemize}
  \item Shall ignore the reaction time required to apply any choices
    made by the cox.
  \item Shall assume the cox has a full 360\textdegree (so vision is
    not blocked by his or other boats). 
  \item Shall assume his
    vision is blocked by the river bank which sits sufficiently high
    above the river to make it impossible to see around corners.
  \item Shall assume the cox can guess reasonable well the speed and
    direction of boats within his line of sight (including his own).
  \item Shall assume the cox has a fairly good memory of the river's
    layout so can guess his location reasonably well from the nearby
    landmarks as mentioned in \ref{riversubsec}.
  \item Shall treat spinning as an atomic action.
\end{itemize}

\section{Plan}

\subsection{Modelling}

Section \ref{prog:model} shows the state of the current
model. However, there are still some questions that must be
answered. For a single boat learning to navigate an empty river it will be necessary to decide what speeds it can
move at (both as desired by an outing plan and maximum) and what rates
turning are available to it. Later on when more boats are introduced
it might be nice to try and simulate traffic conditions similar to
those found in the real world and
so it will be necessary to produce a model of how frequently boats
launch and the distribution of speeds and outing plans across boats,
along with any dependencies (for example, I would guess that boats
with a slower maximum speed tend to go out less frequently and have
shorter outings when they do go out).

\subsection{Implementation}
The implementation plan has two parts. Firstly, process of converting
the model into computer code must be broken down into smaller
tasks. This stage will also include which tools to use to complete the
tasks such as which language and pre-existing libraries or frameworks
to use. This is the plan that lays out what needs doing. Secondly, the
plan should contain a brief description of the manner in which I plan
to work. This can be considered as a brief description of software
engineering methodology. This second part can be expanded now as it
comes from my experience of past projects.

\subsubsection{Software Engineering Methodology}
This project is about more than producing a piece of software but I
consider it to include a significant portion to be software
engineering.  As such I will layout briefly how I plan to follow good
practices. 

\begin{description}
  \item[Version Control] I shall keep my
code (including write-ups like this report) under version control
using Git. Git is very useful to use on individual projects since it
does not require setting up a server for repositories unlike older
version control systems such as Subversion. Using Git also allows me
to store my code on Github. Distribution of my project's repositories
has limited use for an individual project but there are still
occassions when I may wish to access the code when away from my
development machine. It also allows my supervisor (and indeed anyone
else) to follow my progress should she wish to (\url{https://github.com/henryaddison}).
For instance the source for this \LaTeX  report can be found at
\url{https://github.com/henryaddison/msc-background-report}. 

  \item[Iterative] Although the full Agile methodology does not a appropriate to a single-person
project, I still plan to work in an iterative, incremental fashion and
focus on solving one problem at a time, breaking up a complex
task into many, simpler ones. 

  \item[Test Driven Developement] Experience has taught me in the long run, writing
tests is time well spent as it helps to catch and prevent bugs and
also the tests double as a form of living documentation. 

\end{description}

Three months is a very short time so if I do not complete all the goals of the project in
that time I hope these practices will help me write code that I or someone else
could continue to work with at a later date.

\subsection{Evaluation}
In order to evaluate any results produced, it is necessary to decide
first what should be considered as successful outcomes and then how
these should be measured. For example, from a safety point of view it
would be desireable to cutdown on dangerous crashes, which might be
measured as the number of collisions above a certain speed.
