\chapter{Statement of Problem}

Before implementation can begin some time should be taken to carefully define the problem.

\section{Motivation}

As presented in the Introduction the aim of a simulation of conditions
on the river Cam is to try and answer the question ``What should an
inexperienced cox try to do?" There are also other questions that a simulation could help answer.

\begin{enumerate}
  \item What is the pattern of traffic on the river? Why does it get busy in certain parts of the parts at certain times? Could adjustments in cox behaviour improve traffic flow?
  \item Could adjustments to cox behaviour improve safety on the river?
  \item \label{CoopQ} Should individual coxes behave selfishly or would co-operation with other coxes help in the long run (even at the expense of occasional self-sacrifice)?
  \item \label{CommQ} Could communication between coxes be used to improve quality of outings for crews?
\end{enumerate}

It would be interesting to follow up some of these questions as well during the project, particularly questions \ref{CoopQ} and \ref{CommQ}.

\section{Issues facing a cox}
The cox has many issues to deal with in the course of an outing. These
have been split into two types. There are external issues that place
physical constraints on where the boat should be and internal issues
that may affect the result of actions taken by a cox. 


\subsection{External Issues}
\begin{itemize}
  
  \item{\textbf{Navigation}} The cox controls the rudder and dictates the power applied by each rower and so it is his responsibility to steer a path along the river.
  
  \item{\textbf{Congestion}} As well as avoiding static obstacles like
    the river bank, the cox will also have to contend with other river users who may get in his way. The Cam is a very small river with a large number of rowers so dealing with congestion is a common problem.
  
  \item{\textbf{Safety}} The cox is responsible for the safety of the
    boat and crew because he is the person in charge of the boat's
    speed and direction and the one person looking forward in the
    boat. It is the cox's responsibility that no harm comes to people
    or equipment during an outing.
  
  \item{\textbf{Covering distance}} A crew will not get better at racing if they only travel a short distance at a slow speed. Often a coach will supply the crew with an outing plan that will dictate how fast and how far the crew should travel during the outing. The cox should ensure that the crew's actual activity matches the outing plan as closely as possible.
  
\end{itemize}

\subsection{Internal Issues}
\begin{itemize}
  \item{\textbf{Indirect Control}} Unlike a car where the driver can directly adjust speed with the accelerator pedal, a cox can only request that the rowers apply more or less pressure. It is the rowers themselves who must make these changes. Even if these requests are followed only they can only specify very rough changes in pressure applied which in turn means speed can only be altered in hard-to-predict jumps.
  
  \item{\textbf{Physiological Constraints}} \label{physiocons} Each rower is a human and so there are complex physiological (and psychological) constraints that will limit how quickly the rowers can accelerate the boat and how long they can maintain a particular speed.
\end{itemize}

For the most part this project will look at having autonomous coxes
learn to deal with the external issues and not worry too much about
the complications that caused by the internal issues. In particular,
although coaching and motivation are an important part of a cox's
duties, this project will stick to examining the physical issues of maneuvering a boat.

\section{Objects in Problem Domain}\label{sop:objuni}

To make it feasible to create a simulation the model must simplify the real world. The first simplification is to restrict what objects shall be modelled. For the purposes of this project there are four main types of object to consider: the river, boats, outing plans and coxes. 

\subsection{River} \label{riversubsec}
Most rowing on the Cam takes places in the 5km between Jesus Lock and
Baits Bite Lock. A few crews sometimes carry
their boats over Baits Bite Lock to another stretch of river up to
Bottisham lock. The boathouses used by rowers are clustered near Jesus
Lock. The river has been known to flood but for the most
part it is static and its stream has no effect on boats. The river bank is
varied and there are many landmarks
such as corners, bridges, trees, and buildings by the river.

\subsection{Boat}
The river is used by rowing boats
of various sizes both coxed and coxless. The nature of
college racing means that the vast majority of rowing is done in
eights. A commonly used boat is a Janousek eight which is roughly 17m long, 60cm wide and 40cm deep
\cite{Janousek}. The effective width of an eight however is much wider
since we must take into account of the oars. An oar on a rowing boat
is roughly 370cm long \cite{Concept2}, although about 60cm of that
will overlap with the boat itself. Therefore the effective width of an
eight is more like 7m. The rowers can draw in their oars most of the
way to reduce the width of the boat, but this should only be done on
one side and when the boat is stationary if the rowers do not wish to
capsize.

The rowers sit in the boat. The speed and acceleration of a boat
is limited by the strength and skill of the rowers along with
physiological constraints mentioned in \ref{physiocons}. It should also be noted that
not all boats are equal. Male rowers can apply more power than female
rowers. Crews for a boats are selected by ability as well as gender so a 1st men's boat will
be able to move faster than a women's 4th boat.

The river is also used by a few barges and riverboats. Many of these
are lived in and never move from their moorings on the bank by the boathouses. A few do occassionally move
up and down the river. It should be noted that these boats are well
handled. The tourist traffic which behaves more erratically is
restricted to its own, separate part of the river.

\subsection{Outing Plan}
An outing plan is usually a set of exercises to be done during an
outing. It is set before the outing and is part of a training plan aimed
to improve the power and skill of rowers in a boat over a period of time. These can take many forms but in terms of navigation they boil
down to specifying a certain speed (albeit through proxies like
how much pressure should be applied by how many rowers) for a certain
time or distance limit. Outing plans contain overall
constraints like covering a certain distance or being on the water
for a certain amount of time.

\subsection{Cox}
The cox is a the main brains of a boat from a navigation view
point. This is not to denigrate the rowers who must perform very
complex actions in order to propel a boat and must respond to a cox's
orders, but the job of navigation
is delegated to the cox. Therefore for the purposes of this project we
shall consider the cox as an autonomous agent who can make observations about the
river, their boat and other boats and perform actions affecting the
boat's speed, through requests to the rowers, and rate of turning, by
adjusting the rudder.

Alterations to the boats speed can be done in two ways:
either by adjusting the power applied by the rowers or by getting the
rowers to stop rowing and start braking (either in a controlled, comfortable manner
or by making an emergency stop). A boat can also be moved very
slowly backwards from a stationary state. When the boat is stopped the
cox can have the rowers on one side row forwards and those on the other
to row backwards as a way of spinning the boat to face the other
direction. Although often accompanied by a coach on the bank, the cox
has the final word in what the boat does and so for the purposes of
this project it is up to the cox to follow the outing plan
as closely as possible.
