\chapter{Statement of Problem}

Before implementation can begin some time should be taken to carefully define the problem.

\section{Issues facing a cox}
The cox has many issues to deal with in the course of an outing. These have been split into two types - external issues that place physical constraints on where the boat should be and internal issues that may affect the effect of actions taken by a cox. For the most part this project will look at having autonomous coxes learn to deal with the external issues and not worry too much about the complications that the internal issues may introduce.

\subsection{External Issues}
\begin{itemize}
  
  \item{Navigation} The cox controls the rudder and dictates the power applied by each rower and so it is his responsibility to steer a path along the river.
  
  \item{Congestion} As well as avoiding static obstacles, the cox will also have to contend with other river users who may get in his way. The Cam is a very small river with a large number of rowers so dealing with congestion is a common problem.
  
  \item{Safety} The cox is responsible for the safety of the boat and crew because he is the person in charge of the boat's speed and direction and the one person looking forward in the boat. It is the cox's responsibility that the boat does not collide with any obstacles that may cause damage to any people or equipment on the river.
  
  \item{Covering distance} A crew will gain very little from an outing if they only travel a short distance at a slow speed. Often a coach will supply the crew with an outing plan that will dictate how fast and how far the crew should travel during the outing. The cox should ensure that the crew's actual activity matches the outing plan as closely as possible.
  
\end{itemize}

\subsection{Internal Issues}
\begin{itemize}
  \item{Indirect Control} Unlike a car where the driver can directly adjust speed with the accelerator pedal, a cox can only request that the rowers apply more or less pressure. It is the rowers themselves who must make these changes. Even if these requests are followed only they can only specify very rough changes in pressure applied which in turn means speed can only be altered in hard-to-predict jumps.
  
  \item{Physiological Constraints} Each rower is a human and so there are complex physiological (and psychological) constraints that will limit how quickly the rowers can accelerate the boat and how long they can maintain a particular speed.
\end{itemize}


\section{Questions that simulation could help answer}

As presented in the Introduction the aim of a simulation of conditions on the river Cam is to try and answer the question ``What should an inexperienced cox try to do?" There are also other questions that may be of interest that perhaps the simulation could help address.

\begin{enumerate}
  \item What is the pattern of traffic on the river? Why does it get busy in certain parts of the parts at certain times? Could adjustments in cox behaviour improve traffic flow?
  \item Could adjustments to cox behaviour improve safety on the river?
  \item \label{CoopQ} Should individual coxes behave selfishly or would co-operation with other coxes help in the long run (even at the expense of occasional self-sacrifice)?
  \item \label{CommQ} Could communication between coxes be used to improve quality of outings for crews?
\end{enumerate}

It would be interesting to follow up some of these questions as well during the project, particularly questions \ref{CoopQ} and \ref{CommQ}.

\section{Objects in Universe}

To make it feasible to create a simulation the model must simplify the real world. The first simplification is to restrict what objects shall be modelled. For the purposes of this project there are four main types of object to consider: the river, boats, outing plans and coxes. 

\subsection{River} \label{riversubsec}
The vast majority of rowing on the Cam takes places on a 5km stretch
of river between Jesus Lock and Baits Bite Lock. Some crews do carry
their boats over Baits Bite Lock to another stretch of river up to
Bottisham lock. The boathouses used by rowers are clustered near Jesus
Lock. This stretch of the Cam has been known to flood but for the most
part it is static and its stream has no effect on boats. The bank is
very varied along this stretch of river and there are many landmakrs
in the form of corners, bridges, make up of the bank (concrete, grass
or trees, for example) and buildings, fences and other constructions.

\subsubsection{Simplifications}
\begin{itemize}
  \item A crew carrying their boat over Baits Bite Lock happens infrequently enough that it can safely be ignored.
  \item The boathouses are sufficiently close together and near Jesus Lock that it is reasonable to merge the boathouses into one. 
  \item Outside of race days and a brief period early in the year when
    novice crews take to the water, the short stretch of water from
    Christ's boathouse (the last in the row) to Jesus Lock is rarely
    used and so it seems reasonable to treat this merged boathouse as
    one end point of the river rather than Jesus Lock.
  \item Shall assume there are equally spaced, frequent landmarks
    along both banks of the river.
\end{itemize}

\subsection{Boat}
The stretch of the Cam described above is used heavily by rowing boats
of various sizes both coxed and coxless. However, the nature of
college racing means that the vast majority of rowing is done in
eights. Taking the dimensions of a Janousek eight (a commonly used
make), an eight is roughly 17m long, 60cm wide and 40cm deep
\cite{Janousek}. The effective width of an eight however is much wider
since we must take into account of the oars. An oar on a rowing boat
is roughly 370cm long \cite{Concept2}, although about 60cm of that
will overlap with the boat itself. Therefore the effective width of an
eight is more like 7m. The rowers can draw in their oars most of the
way to reduce the width of the boat, but this should only be done on
one side and when the boat is stationary if the rowers do not wish to
capsize.

The boat also holds the rowers, so this is the object that we consider
to have a speed and a direction. The speed and acceleration of a boat
is limited by the strength and skill of the rowers along with
psyiological constraints mentioned above. It should also be noted that
not all boats are equal. Male rowers can apply more power than female
rowers. Boats are often selected on skill so a college 1st boat will
be able to move faster than a 5th boat.

The river is also used by a few barges and riverboats. Many of these
are lived in and never move from their moorings on the bank for part
of the river by the boathouses. However, a few do occassionally move
up and down the river. It should be noted that these boats are well
handled, the tourist traffic which behaves more erratically is
restricted to its own part of the river through town which rowing
boats cannot use.


\subsubsection{Simplifications}
\begin{itemize}
  \item To begin with shall assume all boats are rowing eights.
  \item Restrict to just looking at the 2D cross-section of where boat
    meets water, so can consider a boat to be a rigid 17m x 7m rectangle for
    the purposes of collision detection.
  \item A boat will have a fixed maximum speed and fixed acceleration
    rates and deceleration rates (to begin with these may even be
    infinite). When a boat is launched these parameters will be
    selected at random from a distribution to be determined.
\end{itemize}

\subsection{Outing Plan}
An outing plan is usually a set of exercises to be done during an
outing. It is set before the outing and makes up a training plan aimed
to improve the power and skill of rowers in a boat. These can take many forms but in terms of navigation they boil
down to specifying a certain speed (albeit often through proxies like
how much pressure should be applied by how many rowers) for a certain
time or distance limit. Outing plans will also contain overall
constraints like covering a certain distance or being out on the water
for a certain amount of time (usual maximal as rowers must be back for lectures).
\subsubsection{Simplifications}
\begin{itemize}
  \item An outing plan will specify a single desired speed to travel at
    and a total distance to cover before returning to the
    boathouse. Each time a boat is launched these parameters will be
    chosen at random from distributeion to be determined.
  \item To begin with shall ignore time constraints.
\end{itemize}

\subsection{Cox}
The cox is a the main brains of a boat from a navigation view
point. This is not to denigrate the rowers who must perform very
complex actions in order to propel a boat and must respond to a cox's
orders, but in a coxed boat it is accepted that the job of navigation
is delegated to the cox. Therefore for the purposes of this project we
shall consider the cox as an agent who can make observations about the
river, their boat and other boats. The cox can request alterations to
a boat's speed and, by adjusting the rudder, can alter the rate of
turning. Alterations to the boats speed can be done in two ways:
either by adjusting the power applied by the rowers or by getting the
rowers to stop rowing and start braking (either in a controlled, comfortable manner
or by making an emergency stop). A boat can also be moved very
slowly backwards from a stationary state. When the boat is stopped the
cox can have the rowers on one side row forwards and those on the other
to row backwards as a way of spinning the boat to face the other
direction. Although often accompanied by a coach on the bank, the cox
has the final word in what the boat does and so for the purposes of
this project we shall leave it to the cox to follow the training plan
as close as possible.
\subsubsection{Simplifications}
\begin{itemize}
  \item Shall ignore the reaction time required to apply any choices
    made by the cox.
  \item Shall assume the cox has a full 360\textdegree (so vision is
    not blocked by his or other boats). 
  \item Shall assume his
    vision is blocked by the river bank which sits sufficiently high
    above the river to make it impossible to see around corners.
  \item Shall assume the cox can guess reasonable well the speed and
    direction of boats within his line of sight (including his own).
  \item Shall assume the cox has a fairly good memory of the river's
    layout so can guess his location reasonably well from the nearby
    landmarks as mentioned in \ref{riversubsec}.
  \item Shall treat spinning as an atomic action.
\end{itemize}
