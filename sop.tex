\chapter{Statement of Problem}

Before implementation can begin some time should be taken to carefully define the problem.

\section{Issues facing a cox}
The cox has many issues to deal with in the course of an outing. These have been split into to types. There are issues external to the crew and there are issues internal to the crew

\subsection{External Issues}
\begin{itemize}
  
  \item{Navigation} The cox controls the rudder and dictates the power applied by each rower and so it is his responsibility to steer a path along the river.
  
  \item{Congestion} As well as avoiding static obstacles, the cox will also have to contend with other river users who may get in his way. The Cam is a very small river with a large number of rowers so dealing with congestion is a common problem.
  
  \item{Safety} The cox is responsible for the safety of the boat and crew because he is the person in charge of the boat's speed and direction and the one person looking forward in the boat. It is the cox's responsibility that the boat does not collide with any obstacles that may cause damage to any people or equipment on the river.
  
  \item{Covering distance} A crew will gain very little from an outing if they only travel a short distance at a slow speed. Often a coach will supply the crew with an outing plan that will dictate how fast and how far the crew should travel during the outing. The cox should ensure that the crew's actual activity matches the outing plan as closely as possible.
  
\end{itemize}

\subsection{Internal Issues}
\begin{itemize}
  \item{Indirect Control} Unlike a car where the driver can directly adjust speed with the accelerator pedal, a cox can only request that the rowers apply more or less pressure. It is the rowers themselves who must make these changes. Even if these requests are followed only they can only specify very rough changes in pressure applied which in turn means speed can only be altered in hard-to-predict jumps.
  
  \item{Physiological Constraints} Each rower is a human and so there are complex physiological (and psychological) constraints that will limit how quickly the rowers can accelerate the boat and how long they can maintain a particular speed.
\end{itemize}


\section{Questions that simulation could help answer}

As presented in the Introduction the aim of a simulation of conditions on the river Cam is to try and answer the question ``What should an inexperienced cox try to do?" There are also other questions that may be of interest that perhaps the simulation could help address.

\begin{enumerate}
  \item What is the pattern of traffic on the river? Why does it get busy in certain parts of the parts at certain times? Could adjustments in cox behaviour improve traffic flow?
  \item Could adjustments to cox behaviour improve safety on the river?
  \item \label{CoopQ} Should individual coxes behave selfishly or would co-operation with other coxes help in the long run (even at the expense of occasional self-sacrifice)?
  \item \label{CommQ} Could communication between coxes be used to improve quality of outings for crews?
\end{enumerate}

It would be interesting to follow up some of these questions as well during the project, particularly questions \ref{CoopQ} and \ref{CommQ}.
