\chapter{Relevant Literature}

In preparation for this project, I have done some background
reading. I looked at a report of a similar project and looked for research about design and implementation of intelligent agents. 

\section{Similar Project}

A similar project is a study of recreational river boats using
traffic simulation software \cite{Lowry2011}. This was looking at a
different problem, the amount of traffic passing through various parts
of a river at given times, and so their approach of using VISSIM is
not appropriate to this project. For instance, they could safely
separate the river into two lanes of unidirectional traffic (and the
vast majority of traffic was floating downstream) and they used a
particularly simple overtaking model. The VISSIM driver model is also
a stochastic model while the aim of this project is to develop
intelligent agents whose behaviour model can be inspected for rules to
follow. Despite this, the domain being studied was
similar. In particular they discuss the difficulties of calibrating
their simulation given limited resources and how to make guesses
about useful information based on the limited data that could be
collected. They also provide some ideas about how to get hold of data
in order to model the topology of a river (unfortunately it would
appear that extracting it from an OS map is the best available
technique).

\section{Related Research}

The simulation actions and interactions of these coxed boats can follow an Agent-Based Model, so I looked at related research in two areas: research into control mechanisms for
autonomous agents and looking at tools and frameworks for simulating
and visualizing multi-agent systems.

\subsection{Control Mechanisms}

One way to see this project is to treat the cox and a boat as a mobile
robot. The cox makes observations and decides actions while the boat
acts as the embodiment of the cox and the rowers can be
seen as the robot's actuators. Nguyen-Tuong et al. survey the current state
model learning for robot control in \cite{Nguyen-Tuong2011}. A lot of
the techniques covered in this paper are beyond the scope of
this project since the aim is to learn a simple control policy that a
human can understand. However, it is reassuring to see that robot
control is considered very much an open problem and provides
more motivation for treating the coxes as learning agents to remove the
necessity of pre-programming analytic solutions to all scenarios these
coxes may face.

One particular study in having a robot learn to control itself is by
Banzhaf et al. \cite{Banzhaf1997}. In this case the control program
run by a robot is discovered using an evolutionary algorithm. Although the final program may not be accessible to a
human, the fitness functions used in
the evolutionary process to produce programs exhibiting chosen
behaviour (such as obstacle avoidance or wall-following) are
readable. It should also be noted that the authors failed to produce a program that could switch between the
desired behaviours using their genetic programming appraoch. This is something that a cox needs to do as he
switches between different behaviours such as a following the river, overtaking a
slower crew and spinning.

Braitenberg's Vehicles \cite{Braitenberg1986} are compelling
thought experiments into how apparently complex behaviour can observed in
very simple vehicles. Perhaps viewing the boat as a vehicle with
reacting to stimuli in a straight-forward manner is
sufficient, particularly in the case of an empty river a cox can follow a
simple bank-following strategy.

It may also be that the Braitenberg vehicle required to simulate cox
beahviour will be too complicated to be analysed by a human. Nilsson's
teleo-reactive programs \cite{Nilsson1994} provide an
alternative for producing programs that can used used to control a
boat. A big advantage is that they are designed to be human
readable. The cost of this readability is computational efficiency but
improvements in
computational power since 1994 may have rendered this problem
irrelevant. These programs also capture the fact that a robot should
continue in an action until stopped, matching how rowers are taught to
behave
(rowing in the same manner until told otherwise by the cox). It
would also be interesting to follow up the question Nilsson himself
asks in his paper of how such programs might be learnt by an agent.

\subsection{Tools and Frameworks}
In \cite{Gilbert2002} Gilbert and Bankes explain the advantages and
disadvantages of using pre-existing libraries and frameworks for Agent-Based Models rather
than ``rolling your own.'' Using pre-existing libraries frees a
programmer up from re-implementing common
algorithms. However, they also require a programmer to understand the
language they are written in and to work with the built-in
asssumptions used by the original writers. Gilbert and Bankes state
that agent-based modeling tools that match their ideal specifications
do not yet exist but that there is a trend towards better
tools becoming available.

Allen \cite{Allan2009} surveys a large number of agent based
modelling and simulation tools. Comparing his reviews with those by
Berryman \cite{Berryman2008}, suggest that NetLogo and MASON are two
worth investigating further. NetLogo requires the use of a proprietary
language to program the agents aimed at beginner programmers, which I
found too restrictive. NetLogo offers automatic drawing of agents in
2D and 3D but I found the graphics options insufficiently
sophisticated.

The visualization problem may be got round by using specialized
toolkits such as Visualization Toolkit (VTK) \cite{kitware}. This open source framework
appears sufficiently powerful and adaptable enough that it could be
used in conjunction with a separate framework for simulating the
behaviour of boats.
