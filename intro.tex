\chapter{Introduction}
Casual viewers of rowing during the Olympics may not even see a cox
since only 2 of the 14 rowing events that take place are
coxed. 
However, even for these races on a straight course with each boat in its own lane,
coxes are still included in the men's and women's eights.
In another highly viewed race, the Oxford-Cambridge Boat Race, coxes
are critical as they must carefully steer along
the narrow stream of faster water. In many ways these races are simpler for a cox
than training on a narrow, busy river such as the Cam.

A cox of a college boat on the Cam must deal with coordinating,
coaching and motivating his crew just like the cox of an eight racing
on the rowing lake in the Olympics. He must steer along a narrow and
winding path just like his Boat Race counterparts. On top of this he
must also avoid colliding with the many other boats on the river
moving in different directions and at different speeds. He will also
have a coach on the bank shouting at him to follow a training plan
which dictates how fast and how far the boat should travel. And he
must ensure the crew is back for their next lecture with only the rudder to steer the boat and a few commands to the rowers to adjust the speed of the boat. For a new cox this is a lot to deal with. Wouldn't it be great if there were a few, simple rules of thumb he could follow?

The aim of this project is to simulate these conditions on the river Cam representing the crews as autonomous software agents. These agents will then attempt to learn a protocol that a new cox could follow in order to help a crew make the most of its time on the river. To help verify the correctness of the simulation another aim is to create some visualizations for these simulations.

This background report will examine the problem a little more
carefully in the next section. It will then look at some similar work
that might be involved in solving the problem. Finally it will summarise the work that has been done so far and the plan to extend it.
