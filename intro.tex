\chapter{Introduction}
Casual viewers of rowing during the Olympics may not even see a cox since only 2 of the 14 rowing events that take place are coxed. However, that the men's and women's eights races are still coxed suggests that even on a straight course with each boat having its own lane the cox is not worthless. Indeed consider another highly watched rowing race, the Oxford-Cambridge Boat Race. Coxes are critical in this race since careful steering to stay in the narrow stream of faster water is an important part of this race. The stakes may be higher but even this race is a simpler for a cox than training on a narrow, busy river such as the Cam.

A cox of a college boat on the Cam must deal with coordinating, coaching and motivating his crew just like the cox of an eight racing on the rowing lake in the Olympics. He must steer along a narrow and winding path just like his Boat Race counterparts. But he must also avoid colliding with the many other boats on the river moving in different directions and at different speeds. He will also have a coach on the bank shouting at them to follow a training plan which dictate how fast and how far the boat should travel. And all this must before the crew has to be back for their next lecture with only the rudder to steer the boat and a few commands to the rowers to adjust the speed of the boat. For a new cox this is a lot to deal with. Wouldn't it be great if there were a few, simple rules of thumb he could follow?

The aim of this project is to simulate these conditions on the river Cam representing the crews as autonomous software agents. These agents will then attempt to learn a protocol that a new cox could follow in order to help a crew make the most of its time on the river. To help verify the correctness of the simulation another aim is to create some visualizations for these simulations.